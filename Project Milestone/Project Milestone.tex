\documentclass{article}

% if you need to pass options to natbib, use, e.g.:
% \PassOptionsToPackage{numbers, compress}{natbib}
% before loading nips_2018

% ready for submission
\usepackage{nips_2018}

% to compile a preprint version, e.g., for submission to arXiv, add
% add the [preprint] option:
% \usepackage[preprint]{nips_2018}

% to compile a camera-ready version, add the [final] option, e.g.:
% \usepackage[final]{nips_2018}

% to avoid loading the natbib package, add option nonatbib:
% \usepackage[nonatbib]{nips_2018}

\usepackage[utf8]{inputenc} % allow utf-8 input
\usepackage[T1]{fontenc}    % use 8-bit T1 fonts
\usepackage{hyperref}       % hyperlinks
\usepackage{url}            % simple URL typesetting
\usepackage{booktabs}       % professional-quality tables
\usepackage{amsfonts}       % blackboard math symbols
\usepackage{nicefrac}       % compact symbols for 1/2, etc.
\usepackage{microtype}      % microtypography

\title{Project Milestone}

% The \author macro works with any number of authors. There are two
% commands used to separate the names and addresses of multiple
% authors: \And and \AND.
%
% Using \And between authors leaves it to LaTeX to determine where to
% break the lines. Using \AND forces a line break at that point. So,
% if LaTeX puts 3 of 4 authors names on the first line, and the last
% on the second line, try using \AND instead of \And before the third
% author name.

\author{ M.P.Ross}

\begin{document}

\maketitle

\begin{abstract}

\end{abstract}

\section{Introduction}

\section{Synthetic event creation}

\section{First attempts at classifier}

\section{Future investigations}


\begin{figure}
  \centering
  \fbox{\rule[-.5cm]{0cm}{4cm} \rule[-.5cm]{4cm}{0cm}}
  \caption{Sample figure caption.}
\end{figure}


\section*{References}

\medskip

\small

[1] Gossan, S.E., Sutton, P., Sutver, A., Zanolin, M., Gill, K., \ \& Ott, C.D\ (2016) Observing gravitational waves from core-collapse supernovae in the advanced detector era. {\it Phys. Rev. D} {\bf 93}, 042002

[2] George, D., \ \& Huerta, E.A\ (2018) 
Deep Learning for real-time gravitational wave detection and parameter estimation: Results with Advanced LIGO data. {\it Physics Letters B} {\bf 778}

[3] George, D., \ \& Huerta, E.A\ (2018) Deep neural networks to enable real-time multimessenger astrophysics. {\it Phys. Rev. D} {\bf 97}, 044039

\end{document}